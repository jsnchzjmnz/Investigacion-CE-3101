%%%%%%%%%%%%%%%%%%%%%%%%%%%%%%%%%%%%%%%%%
%Disenio basado en el template:
%
% Large Colored Title Article
% LaTeX Template
% Version 1.1 (25/11/12)
%
% This template has been downloaded from:
% http://www.LaTeXTemplates.com
%
% Original author:
% Frits Wenneker (http://www.howtotex.com)
%
% License:
% CC BY-NC-SA 3.0 (http://creativecommons.org/licenses/by-nc-sa/3.0/)
%
%%%%%%%%%%%%%%%%%%%%%%%%%%%%%%%%%%%%%%%%%

%----------------------------------------------------------------------------------------
%	PACKAGES AND OTHER DOCUMENT CONFIGURATIONS
%----------------------------------------------------------------------------------------

\documentclass[DIV=calc, paper=a4, fontsize=11pt, twocolumn]{scrartcl}	 % A4 paper and 11pt font size

\usepackage{lipsum} % Used for inserting dummy 'Lorem ipsum' text into the template
\usepackage[english]{babel} % English language/hyphenation
\usepackage[protrusion=true,expansion=true]{microtype} % Better typography
\usepackage{amsmath,amsfonts,amsthm} % Math packages
\usepackage[svgnames]{xcolor} % Enabling colors by their 'svgnames'
\usepackage[hang, small,labelfont=bf,up,textfont=it,up]{caption} % Custom captions under/above floats in tables or figures
\usepackage{booktabs} % Horizontal rules in tables
\usepackage{fix-cm}	 % Custom font sizes - used for the initial letter in the document

\usepackage{sectsty} % Enables custom section titles
\allsectionsfont{\usefont{OT1}{phv}{b}{n}} % Change the font of all section commands

\usepackage{fancyhdr} % Needed to define custom headers/footers
\pagestyle{fancy} % Enables the custom headers/footers
\usepackage{lastpage} % Used to determine the number of pages in the document (for "Page X of Total")

% Headers - all currently empty
\lhead{}
\chead{}
\rhead{}

% Footers
\lfoot{}
\cfoot{}
\rfoot{\footnotesize Page \thepage\ of \pageref{LastPage}} % "Page 1 of 2"

\renewcommand{\headrulewidth}{0.0pt} % No header rule
\renewcommand{\footrulewidth}{0.4pt} % Thin footer rule

\usepackage{lettrine} % Package to accentuate the first letter of the text
\newcommand{\initial}[1]{ % Defines the command and style for the first letter
\lettrine[lines=3,lhang=0.3,nindent=0em]{
\color{DarkGoldenrod}
{\textsf{#1}}}{}}

%----------------------------------------------------------------------------------------
%	TITLE SECTION
%----------------------------------------------------------------------------------------

\usepackage{titling} % Allows custom title configuration

\newcommand{\HorRule}{\color{DarkGoldenrod} \rule{\linewidth}{1pt}} % Defines the gold horizontal rule around the title

\pretitle{\vspace{-30pt} \begin{flushleft} \HorRule \fontsize{50}{50} \usefont{OT1}{phv}{b}{n} \color{DarkRed} \selectfont} % Horizontal rule before the title

\title{Una Cucharadita de Bases de Datos Multidimensionales} % Your article title

\posttitle{\par\end{flushleft}\vskip 0.5em} % Whitespace under the title

\preauthor{\begin{flushleft}\large \lineskip 0.5em \usefont{OT1}{phv}{b}{sl} \color{DarkRed}} % Author font configuration

\author{Julio S\'{a}nchez,Maikol Barrantes,Frander Granados } % Your name

\postauthor{\footnotesize \usefont{OT1}{phv}{m}{sl} \color{Black} % Configuration for the institution name
Tecnol\'{o}gico de Costa Rica % Your institution

\par\end{flushleft}\HorRule} % Horizontal rule after the title

\date{29/09/2015} % Add a date here if you would like one to appear underneath the title block

%----------------------------------------------------------------------------------------

\begin{document}

\maketitle % Print the title

\thispagestyle{fancy} % Enabling the custom headers/footers for the first page 

%----------------------------------------------------------------------------------------
%	ABSTRACT
%----------------------------------------------------------------------------------------

% The first character should be within \initial{}
\initial{D}\textbf{atabases nowadays are one of the most important tools in our day by day labours, and it use is fundamental in all kind of activities that include the manipulation and store of data.\\
In Databases world, there are a lot of specifications. One of them is focused in the analysis of data, and not in transactions or any other kind of process. For this needs, Multidimensional databases are a great tool, that allow to analyse data according to some patterns that include different parameters of searching.}

%----------------------------------------------------------------------------------------
%	ARTICLE CONTENTS
%----------------------------------------------------------------------------------------


%------------------------------------------------
%Introducción ¿Qué es una base de datos Multidimensional?
%------------------------------------------------
\section*{Introducci\'{o}n ¿Qu\'{e} es una Base de Datos Multidimensional?}

\lipsum[1-3] % Dummy text

\begin{align}
A = 
\begin{bmatrix}
A_{11} & A_{21} \\
A_{21} & A_{22}
\end{bmatrix}
\end{align}

\lipsum[4] % Dummy text


%------------------------------------------------
%Modelo Multidimensional de Base de Datos
%------------------------------------------------

\section*{Modelo Multidimensional de Base de Datos}

\lipsum[8] % Dummy text

\begin{description}
\item[First] This is the first item
\item[Last] This is the last item
\end{description}

\lipsum[9] % Dummy text

%------------------------------------------------

\subsection*{Bases de Datos Multidimensionales}

\lipsum[5] % Dummy text

\begin{itemize}
\item First item in a list 
\item Second item in a list 
\item Third item in a list
\end{itemize}

\lipsum[6] % Dummy text

%------------------------------------------------

\subsection*{Data Warehouse}

\lipsum[5] % Dummy text

\begin{itemize}
\item First item in a list 
\item Second item in a list 
\item Third item in a list
\end{itemize}

\lipsum[6] % Dummy text

%------------------------------------------------

\subsection*{Esquemas Data Warehouse}

\lipsum[5] % Dummy text

\begin{itemize}
\item First item in a list 
\item Second item in a list 
\item Third item in a list
\end{itemize}

\lipsum[6] % Dummy text

%------------------------------------------------

\subsection*{Procesos y Medidas de Negocio}

\lipsum[5] % Dummy text

\begin{itemize}
\item First item in a list 
\item Second item in a list 
\item Third item in a list
\end{itemize}

\lipsum[6] % Dummy text


%------------------------------------------------
%Diseño de análisis OLAP
%------------------------------------------------
\section*{Dise\~{n}o de an\'{a}lisis OLAP}

\lipsum[1-3] % Dummy text

\begin{align}
A = 
\begin{bmatrix}
A_{11} & A_{21} \\
A_{21} & A_{22}
\end{bmatrix}
\end{align}

\lipsum[4] % Dummy text

%------------------------------------------------

\subsection*{Qu\'{e} es OLAP?}

\lipsum[5] % Dummy text

\begin{itemize}
\item First item in a list 
\item Second item in a list 
\item Third item in a list
\end{itemize}

\lipsum[6] % Dummy text

%------------------------------------------------

\subsection*{Tipos de OLAP}

\lipsum[7] % Dummy text

\begin{table}
\caption{Random table}
\centering
\begin{tabular}{llr}
\toprule
\multicolumn{2}{c}{Name} \\
\cmidrule(r){1-2}
First name & Last Name & Grade \\
\midrule
John & Doe & $7.5$ \\
Richard & Miles & $2$ \\
\bottomrule
\end{tabular}
\end{table}

%------------------------------------------------

\subsection*{Elementos de OLAP}

\lipsum[7] % Dummy text

\begin{table}
\caption{Random table}
\centering
\begin{tabular}{llr}
\toprule
\multicolumn{2}{c}{Name} \\
\cmidrule(r){1-2}
First name & Last Name & Grade \\
\midrule
John & Doe & $7.5$ \\
Richard & Miles & $2$ \\
\bottomrule
\end{tabular}
\end{table}

%------------------------------------------------

\subsection*{Reglas OLAP de E. F. Codd}

\lipsum[7] % Dummy text

\begin{table}
\caption{Random table}
\centering
\begin{tabular}{llr}
\toprule
\multicolumn{2}{c}{Name} \\
\cmidrule(r){1-2}
First name & Last Name & Grade \\
\midrule
John & Doe & $7.5$ \\
Richard & Miles & $2$ \\
\bottomrule
\end{tabular}
\end{table}

\subsection*{MDX (Multidimensional Query eXpressions) o el SQL de OLAP}

\lipsum[7] % Dummy text

\begin{table}
\caption{Random table}
\centering
\begin{tabular}{llr}
\toprule
\multicolumn{2}{c}{Name} \\
\cmidrule(r){1-2}
First name & Last Name & Grade \\
\midrule
John & Doe & $7.5$ \\
Richard & Miles & $2$ \\
\bottomrule
\end{tabular}
\end{table}
%------------------------------------------------
%Introducción al Modelo LC
%------------------------------------------------
\section*{Introducci\'{o}n al Modelo LC}

\lipsum[1-3] % Dummy text

\begin{align}
A = 
\begin{bmatrix}
A_{11} & A_{21} \\
A_{21} & A_{22}
\end{bmatrix}
\end{align}

\lipsum[4] % Dummy text

%------------------------------------------------

\subsection*{Metadata gen\'{e}rica y no gen\'{e}rica}

\lipsum[5] % Dummy text

\begin{itemize}
\item First item in a list 
\item Second item in a list 
\item Third item in a list
\end{itemize}

\lipsum[6] % Dummy text

%------------------------------------------------

\subsection*{Jerarqu\'{i}a como parte intr\'{i}nseca de una dimensi\'{o}n}

\lipsum[7] % Dummy text

\begin{table}
\caption{Random table}
\centering
\begin{tabular}{llr}
\toprule
\multicolumn{2}{c}{Name} \\
\cmidrule(r){1-2}
First name & Last Name & Grade \\
\midrule
John & Doe & $7.5$ \\
Richard & Miles & $2$ \\
\bottomrule
\end{tabular}
\end{table}

%------------------------------------------------

\subsection*{Qu\'{e} constituye un cubo?}

\lipsum[5] % Dummy text

\begin{itemize}
\item First item in a list 
\item Second item in a list 
\item Third item in a list
\end{itemize}

\lipsum[6] % Dummy text
%------------------------------------------------

\subsection*{C\'{o}mo interpretar las células aisladas?}

\lipsum[5] % Dummy text

\begin{itemize}
\item First item in a list 
\item Second item in a list 
\item Third item in a list
\end{itemize}

\lipsum[6] % Dummy text
%------------------------------------------------

\subsection*{D\'{o}nde pertenecen las f\'{o}rmulas?}

\lipsum[5] % Dummy text

\begin{itemize}
\item First item in a list 
\item Second item in a list 
\item Third item in a list
\end{itemize}

\lipsum[6] % Dummy text
%------------------------------------------------
%Estructura interna de una dimensión
%------------------------------------------------
\section*{Estructura interna de una dimensi\'{o}n}

\lipsum[1-3] % Dummy text

\begin{align}
A = 
\begin{bmatrix}
A_{11} & A_{21} \\
A_{21} & A_{22}
\end{bmatrix}
\end{align}

\lipsum[4] % Dummy text

%------------------------------------------------

\subsection*{Aspectos base no jerárquicos}

\lipsum[5] % Dummy text

\begin{itemize}
\item M\'{e}todo de definici\'{o}n
\item Ordenamiento 
\item M\'{e}tricas
\end{itemize}

\lipsum[6] % Dummy text

%------------------------------------------------

\subsection*{Tipos de estructuras jer\'{a}rquicas}

\lipsum[7] % Dummy text

\begin{table}
\caption{Random table}
\centering
\begin{tabular}{llr}
\toprule
\multicolumn{2}{c}{Name} \\
\cmidrule(r){1-2}
First name & Last Name & Grade \\
\midrule
John & Doe & $7.5$ \\
Richard & Miles & $2$ \\
\bottomrule
\end{tabular}
\end{table}

%------------------------------------------------
%Cubo OLAP en SQL Server
%------------------------------------------------
\section*{Cubo OLAP en SQL Server}

\lipsum[1-3] % Dummy text

\begin{align}
A = 
\begin{bmatrix}
A_{11} & A_{21} \\
A_{21} & A_{22}
\end{bmatrix}
\end{align}

\lipsum[4] % Dummy text

%------------------------------------------------

\subsection*{C\'{o}mo crear un cubo OLAP en SQL Server?}

\lipsum[5] % Dummy text

\begin{itemize}
\item First item in a list 
\item Second item in a list 
\item Third item in a list
\end{itemize}

\lipsum[6] % Dummy text

%------------------------------------------------

\subsection*{Consultas al cubo OLAP}

\lipsum[7] % Dummy text

\begin{table}
\caption{Random table}
\centering
\begin{tabular}{llr}
\toprule
\multicolumn{2}{c}{Name} \\
\cmidrule(r){1-2}
First name & Last Name & Grade \\
\midrule
John & Doe & $7.5$ \\
Richard & Miles & $2$ \\
\bottomrule
\end{tabular}
\end{table}

%------------------------------------------------

\subsection*{Consultas MDX vs Consultas SQL}

\lipsum[5] % Dummy text

\begin{itemize}
\item First item in a list 
\item Second item in a list 
\item Third item in a list
\end{itemize}

\lipsum[6] % Dummy text

%------------------------------------------------

\subsection*{Cubo OLAP vs Base de datos Relacional}

\lipsum[5] % Dummy text

\begin{itemize}
\item First item in a list 
\item Second item in a list 
\item Third item in a list
\end{itemize}

\lipsum[6] % Dummy text


%------------------------------------------------
%HiperCubos
%------------------------------------------------
\section*{Hipercubos}

\lipsum[1-3] % Dummy text

\begin{align}
A = 
\begin{bmatrix}
A_{11} & A_{21} \\
A_{21} & A_{22}
\end{bmatrix}
\end{align}

\lipsum[4] % Dummy text


%------------------------------------------------
%Conclusiones
%------------------------------------------------
\section*{Conclusiones}

\lipsum[1-3] % Dummy text

\begin{align}
A = 
\begin{bmatrix}
A_{11} & A_{21} \\
A_{21} & A_{22}
\end{bmatrix}
\end{align}

\lipsum[4] % Dummy text

%------------------------------------------------

\subsection*{Cuando usar una base de datos Multidimensional?}

\lipsum[5] % Dummy text

\begin{itemize}
\item First item in a list 
\item Second item in a list 
\item Third item in a list
\end{itemize}

\lipsum[6] % Dummy text

%------------------------------------------------

\subsection*{Ventajas y Desventajas de las bases de datos Multidimensionales}

\lipsum[7] % Dummy text

\begin{table}
\caption{Random table}
\centering
\begin{tabular}{llr}
\toprule
\multicolumn{2}{c}{Name} \\
\cmidrule(r){1-2}
First name & Last Name & Grade \\
\midrule
John & Doe & $7.5$ \\
Richard & Miles & $2$ \\
\bottomrule
\end{tabular}
\end{table}
%------------------------------------------------

\subsection*{Comparaci\'{o}n entre tipos de OLAP}

\lipsum[5] % Dummy text

\begin{itemize}
\item First item in a list 
\item Second item in a list 
\item Third item in a list
\end{itemize}

\lipsum[6] % Dummy text

%----------------------------------------------------------------------------------------
%	REFERENCE LIST
%----------------------------------------------------------------------------------------

\begin{thebibliography}{99} % Bibliography - this is intentionally simple in this template

\bibitem[]{Ramakrishnan:2003dg}
Ramakrishnan, R.~ Gehrke, J. (2003).
\newblock SISTEMAS DE GESTION DE ´
BASES DE DATOS (Tercera ed.).
\newblock {\em Espa\~{n}a: McGraw-Hill.}

\bibitem[]{Thomsen:2003dg}
Thomsen, E. (2002).
\newblock OLAP Solutions: Building Multidimentional Information Systems (Segunda ed.).
\newblock {\em New York: John Wiley & Sons.}
 
\end{thebibliography}

%----------------------------------------------------------------------------------------

\end{document}
